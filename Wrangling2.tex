\documentclass[]{tufte-handout}

% ams
\usepackage{amssymb,amsmath}

\usepackage{ifxetex,ifluatex}
\usepackage{fixltx2e} % provides \textsubscript
\ifnum 0\ifxetex 1\fi\ifluatex 1\fi=0 % if pdftex
  \usepackage[T1]{fontenc}
  \usepackage[utf8]{inputenc}
\else % if luatex or xelatex
  \makeatletter
  \@ifpackageloaded{fontspec}{}{\usepackage{fontspec}}
  \makeatother
  \defaultfontfeatures{Ligatures=TeX,Scale=MatchLowercase}
  \makeatletter
  \@ifpackageloaded{soul}{
     \renewcommand\allcapsspacing[1]{{\addfontfeature{LetterSpace=15}#1}}
     \renewcommand\smallcapsspacing[1]{{\addfontfeature{LetterSpace=10}#1}}
   }{}
  \makeatother
\fi

% graphix
\usepackage{graphicx}
\setkeys{Gin}{width=\linewidth,totalheight=\textheight,keepaspectratio}

% booktabs
\usepackage{booktabs}

% url
\usepackage{url}

% hyperref
\usepackage{hyperref}

% units.
\usepackage{units}


\setcounter{secnumdepth}{-1}

% citations

% pandoc syntax highlighting
\usepackage{color}
\usepackage{fancyvrb}
\newcommand{\VerbBar}{|}
\newcommand{\VERB}{\Verb[commandchars=\\\{\}]}
\DefineVerbatimEnvironment{Highlighting}{Verbatim}{commandchars=\\\{\}}
% Add ',fontsize=\small' for more characters per line
\newenvironment{Shaded}{}{}
\newcommand{\KeywordTok}[1]{\textcolor[rgb]{0.00,0.44,0.13}{\textbf{{#1}}}}
\newcommand{\DataTypeTok}[1]{\textcolor[rgb]{0.56,0.13,0.00}{{#1}}}
\newcommand{\DecValTok}[1]{\textcolor[rgb]{0.25,0.63,0.44}{{#1}}}
\newcommand{\BaseNTok}[1]{\textcolor[rgb]{0.25,0.63,0.44}{{#1}}}
\newcommand{\FloatTok}[1]{\textcolor[rgb]{0.25,0.63,0.44}{{#1}}}
\newcommand{\ConstantTok}[1]{\textcolor[rgb]{0.53,0.00,0.00}{{#1}}}
\newcommand{\CharTok}[1]{\textcolor[rgb]{0.25,0.44,0.63}{{#1}}}
\newcommand{\SpecialCharTok}[1]{\textcolor[rgb]{0.25,0.44,0.63}{{#1}}}
\newcommand{\StringTok}[1]{\textcolor[rgb]{0.25,0.44,0.63}{{#1}}}
\newcommand{\VerbatimStringTok}[1]{\textcolor[rgb]{0.25,0.44,0.63}{{#1}}}
\newcommand{\SpecialStringTok}[1]{\textcolor[rgb]{0.73,0.40,0.53}{{#1}}}
\newcommand{\ImportTok}[1]{{#1}}
\newcommand{\CommentTok}[1]{\textcolor[rgb]{0.38,0.63,0.69}{\textit{{#1}}}}
\newcommand{\DocumentationTok}[1]{\textcolor[rgb]{0.73,0.13,0.13}{\textit{{#1}}}}
\newcommand{\AnnotationTok}[1]{\textcolor[rgb]{0.38,0.63,0.69}{\textbf{\textit{{#1}}}}}
\newcommand{\CommentVarTok}[1]{\textcolor[rgb]{0.38,0.63,0.69}{\textbf{\textit{{#1}}}}}
\newcommand{\OtherTok}[1]{\textcolor[rgb]{0.00,0.44,0.13}{{#1}}}
\newcommand{\FunctionTok}[1]{\textcolor[rgb]{0.02,0.16,0.49}{{#1}}}
\newcommand{\VariableTok}[1]{\textcolor[rgb]{0.10,0.09,0.49}{{#1}}}
\newcommand{\ControlFlowTok}[1]{\textcolor[rgb]{0.00,0.44,0.13}{\textbf{{#1}}}}
\newcommand{\OperatorTok}[1]{\textcolor[rgb]{0.40,0.40,0.40}{{#1}}}
\newcommand{\BuiltInTok}[1]{{#1}}
\newcommand{\ExtensionTok}[1]{{#1}}
\newcommand{\PreprocessorTok}[1]{\textcolor[rgb]{0.74,0.48,0.00}{{#1}}}
\newcommand{\AttributeTok}[1]{\textcolor[rgb]{0.49,0.56,0.16}{{#1}}}
\newcommand{\RegionMarkerTok}[1]{{#1}}
\newcommand{\InformationTok}[1]{\textcolor[rgb]{0.38,0.63,0.69}{\textbf{\textit{{#1}}}}}
\newcommand{\WarningTok}[1]{\textcolor[rgb]{0.38,0.63,0.69}{\textbf{\textit{{#1}}}}}
\newcommand{\AlertTok}[1]{\textcolor[rgb]{1.00,0.00,0.00}{\textbf{{#1}}}}
\newcommand{\ErrorTok}[1]{\textcolor[rgb]{1.00,0.00,0.00}{\textbf{{#1}}}}
\newcommand{\NormalTok}[1]{{#1}}

% longtable
\usepackage{longtable,booktabs}

% multiplecol
\usepackage{multicol}

% strikeout
\usepackage[normalem]{ulem}

% morefloats
\usepackage{morefloats}


% tightlist macro required by pandoc >= 1.14
\providecommand{\tightlist}{%
  \setlength{\itemsep}{0pt}\setlength{\parskip}{0pt}}

% title / author / date
\title{Projects in Data Science\\
Activity A2b: Changing Cases}
\author{INSERT STUDENT NAME HERE}


\usepackage{amsthm}
\newtheorem{theorem}{Theorem}
\newtheorem{lemma}{Lemma}
\theoremstyle{definition}
\newtheorem{definition}{Definition}
\newtheorem{corollary}{Corollary}
\newtheorem{proposition}{Proposition}
\theoremstyle{definition}
\newtheorem{example}{Example}
\theoremstyle{definition}
\newtheorem{exercise}{Exercise}
\theoremstyle{remark}
\newtheorem*{remark}{Remark}
\newtheorem*{solution}{Solution}
\let\BeginKnitrBlock\begin \let\EndKnitrBlock\end
\begin{document}

\maketitle




\section{Spread, Gather, and Wide and Narrow Data
Formats}\label{spread-gather-and-wide-and-narrow-data-formats}

\begin{marginfigure}
Additional reading:
\href{http://r4ds.had.co.nz/tidy-data.html\#spreading-and-gathering}{Wickham
and Grolemund on spreading and gathering} or Chapter 11 of Data
Computing by Kaplan
\end{marginfigure}

As we are transforming data, it is important to keep in mind what
constitutes each case (row) of the data. For example, in the initial
\texttt{BabyName} data below, each case is a single name-sex-year
combination. So if we have the same name and sex but a different year,
that would be a different case.

\begin{table}

\caption{\label{tab:unnamed-chunk-2}Each case is one name-sex-year combination.}
\centering
\begin{tabular}[t]{l|l|r|r}
\hline
name & sex & count & year\\
\hline
Mary & F & 7065 & 1880\\
\hline
Anna & F & 2604 & 1880\\
\hline
Emma & F & 2003 & 1880\\
\hline
Elizabeth & F & 1939 & 1880\\
\hline
Minnie & F & 1746 & 1880\\
\hline
Margaret & F & 1578 & 1880\\
\hline
\end{tabular}
\end{table}

It is often necessary to rearrange your data in order to create
visualizations, run statistical analysis, etc. We have already seen some
ways to rearrange the data to change the case. For example, what is the
case after performing the following command?

\begin{Shaded}
\begin{Highlighting}[]
\NormalTok{BabyNamesTotal<-BabyNames %>%}
\StringTok{  }\KeywordTok{group_by}\NormalTok{(name,sex) %>%}
\StringTok{  }\KeywordTok{summarise}\NormalTok{(}\DataTypeTok{total=}\KeywordTok{sum}\NormalTok{(count))}
\end{Highlighting}
\end{Shaded}

Each case now represents one name-sex combination:

\begin{table}

\caption{\label{tab:unnamed-chunk-4}Narrow format where each case is one name-sex combination.}
\centering
\begin{tabular}[t]{l|l|r}
\hline
name & sex & total\\
\hline
Aaban & M & 56\\
\hline
Aabha & F & 12\\
\hline
Aabid & M & 5\\
\hline
Aabriella & F & 5\\
\hline
Aadam & M & 177\\
\hline
Aadan & M & 104\\
\hline
\end{tabular}
\end{table}

In this activity, we are going to learn two new operations to reshape
and reorganize the data: \texttt{spread()} and \texttt{gather()}.

\subsection{Spread}\label{spread}

\BeginKnitrBlock{example}
\protect\hypertarget{exm:unnamed-chunk-5}{}{\label{exm:unnamed-chunk-5} }We
want to find the common names that are the most gender neutral (used
roughly equally for males and females). How should we rearrange the
data? Well, one nice way would be to have a single row for each name,
and then have separate variables for the number of times that name is
used for males and females. Using these two columns, we can then compute
a third column that gives the ratio between these two columns. That is,
we'd like to transform the data into a \textbf{wide format} with each of
the possible values of the \texttt{sex} variable becoming its own
column. The operation we need to perform this transformation is
\textbf{\texttt{spread()}}. It takes a value (\texttt{total} in this
case) representing the variable to be divided into multiple new
variables, and a key (the original variable \texttt{sex} in this case)
that identifies the variable in the initial narrow format data whose
values should become the names of the new variables in the wide format
data. The entry \texttt{fill=0} specifies that if there are, e.g., no
females named Aadam, we should include a zero in the corresponding entry
of the wide format table.
\EndKnitrBlock{example}

\begin{Shaded}
\begin{Highlighting}[]
\NormalTok{BabyWide<-BabyNamesTotal %>%}
\StringTok{  }\KeywordTok{spread}\NormalTok{(}\DataTypeTok{key=}\NormalTok{sex,}\DataTypeTok{value=}\NormalTok{total,}\DataTypeTok{fill=}\DecValTok{0}\NormalTok{)}
\end{Highlighting}
\end{Shaded}

\begin{table}

\caption{\label{tab:unnamed-chunk-7}A wide format with one case per name enables us to examine gender balance.}
\centering
\begin{tabular}[t]{l|r|r}
\hline
name & F & M\\
\hline
Aaban & 0 & 56\\
\hline
Aabha & 12 & 0\\
\hline
Aabid & 0 & 5\\
\hline
Aabriella & 5 & 0\\
\hline
Aadam & 0 & 177\\
\hline
Aadan & 0 & 104\\
\hline
\end{tabular}
\end{table}

Now we can choose common names with frequency greater than 25,000 for
both males and females, and sort by the ratio to identify gender-neutral
names.

\begin{Shaded}
\begin{Highlighting}[]
\NormalTok{Neutral<-BabyWide %>%}
\StringTok{  }\KeywordTok{filter}\NormalTok{(M>}\DecValTok{25000}\NormalTok{,F>}\DecValTok{25000}\NormalTok{) %>%}
\StringTok{  }\KeywordTok{mutate}\NormalTok{(}\DataTypeTok{ratio =} \KeywordTok{pmin}\NormalTok{(M/F,F/M)) %>%}
\StringTok{  }\KeywordTok{arrange}\NormalTok{(}\KeywordTok{desc}\NormalTok{(ratio))}
\end{Highlighting}
\end{Shaded}

\begin{table}

\caption{\label{tab:unnamed-chunk-9}The most gender-neutral common names, in wide format.}
\centering
\begin{tabular}[t]{l|r|r|r}
\hline
name & F & M & ratio\\
\hline
Kerry & 48410 & 49355 & 0.9808530\\
\hline
Riley & 76811 & 85039 & 0.9032444\\
\hline
Jackie & 90217 & 78061 & 0.8652582\\
\hline
Frankie & 32138 & 39662 & 0.8102970\\
\hline
Peyton & 54121 & 42194 & 0.7796234\\
\hline
Jaime & 49406 & 65068 & 0.7592980\\
\hline
Casey & 74699 & 108072 & 0.6911966\\
\hline
Pat & 40124 & 26722 & 0.6659854\\
\hline
Jessie & 165106 & 109031 & 0.6603697\\
\hline
Kendall & 50280 & 32714 & 0.6506364\\
\hline
Jody & 55612 & 30983 & 0.5571280\\
\hline
Avery & 81849 & 44710 & 0.5462498\\
\hline
\end{tabular}
\end{table}

\subsection{Gather}\label{gather}

Next, let's filter these names to keep only those with a ratio of 0.5 or
greater (no more than 2 to 1), and then switch back to narrow format. We
can do this with the following \textbf{\texttt{gather()}} operation. It
gathers the columns listed (\texttt{F},\texttt{M}) at the end into a
single column whose name is given by the key (\texttt{sex}), and
includes the values in a column called \texttt{total}.

\begin{Shaded}
\begin{Highlighting}[]
\NormalTok{NeutralNarrow<-Neutral %>%}
\StringTok{  }\KeywordTok{filter}\NormalTok{(ratio>=.}\DecValTok{5}\NormalTok{) %>%}
\StringTok{  }\KeywordTok{gather}\NormalTok{(}\DataTypeTok{key=}\NormalTok{sex,}\DataTypeTok{value=}\NormalTok{total,F,M)%>%}
\StringTok{  }\KeywordTok{select}\NormalTok{(name,sex,total)%>%}
\StringTok{  }\KeywordTok{arrange}\NormalTok{(name)}
\end{Highlighting}
\end{Shaded}

\begin{table}

\caption{\label{tab:unnamed-chunk-11}Narrow format for the most gender-neutral common names.}
\centering
\begin{tabular}[t]{l|l|r}
\hline
name & sex & total\\
\hline
Avery & F & 81849\\
\hline
Avery & M & 44710\\
\hline
Casey & F & 74699\\
\hline
Casey & M & 108072\\
\hline
Frankie & F & 32138\\
\hline
Frankie & M & 39662\\
\hline
\end{tabular}
\end{table}

\section{Summary Graphic}\label{summary-graphic}

Here is a nice summary graphic of \texttt{gather} and \texttt{spread}
from the
\href{https://github.com/rstudio/cheatsheets/raw/master/source/pdfs/data-import-cheatsheet.pdf}{RStudio
cheat sheet on data import}:

\includegraphics[width=700px]{https://www.macalester.edu/~dshuman1/data/112/fig-spread-gather}

\section{The Daily Show Guests}\label{the-daily-show-guests}

The data associated with
\href{https://fivethirtyeight.com/datalab/every-guest-jon-stewart-ever-had-on-the-daily-show/}{this
article} is available in the \texttt{fivethirtyeight} package, and is
loaded into \texttt{Daily} below. It includes a list of every guest to
ever appear on Jon Stewart's The Daily Show.\footnote{Note that when
  multiple people appeared together, each person receives their own
  line.}

\begin{Shaded}
\begin{Highlighting}[]
\NormalTok{Daily<-daily_show_guests}
\end{Highlighting}
\end{Shaded}

\begin{tabular}{r|l|l|l|l}
\hline
year & google\_knowledge\_occupation & show & group & raw\_guest\_list\\
\hline
1999 & singer & 1999-07-26 & Musician & Donny Osmond\\
\hline
1999 & actress & 1999-07-27 & Acting & Wendie Malick\\
\hline
1999 & vocalist & 1999-07-28 & Musician & Vince Neil\\
\hline
1999 & film actress & 1999-07-29 & Acting & Janeane Garofalo\\
\hline
1999 & comedian & 1999-08-10 & Comedy & Dom Irrera\\
\hline
1999 & actor & 1999-08-11 & Acting & Pierce Brosnan\\
\hline
1999 & director & 1999-08-12 & Media & Eduardo Sanchez and Daniel Myrick\\
\hline
1999 & film director & 1999-08-12 & Media & Eduardo Sanchez and Daniel Myrick\\
\hline
1999 & american television personality & 1999-08-16 & Media & Carson Daly\\
\hline
1999 & actress & 1999-08-17 & Acting & Molly Ringwald\\
\hline
1999 & actress & 1999-08-18 & Acting & Sarah Jessica Parker\\
\hline
\end{tabular}

\subsection{Popular guests}\label{popular-guests}

\BeginKnitrBlock{exercise}
\protect\hypertarget{exr:unnamed-chunk-15}{}{\label{exr:unnamed-chunk-15}
}Create the following table containing 19 columns. The first column
should have the ten guests with the highest number of total apperances
on the show, listed in descending order of number of appearances. The
next 17 columns should show the number of appearances of the
corresponding guest in each year from 1999 to 2015 (one per column). The
final column should show the total number of appearances for the
corresponding guest over the entire duration of the show (these entries
should be in decreasing order).
\EndKnitrBlock{exercise}

\begin{verbatim}
PopDaily <- 
  Daily %>%
  gather(key=raw_guest_list,appearance=n())
PopDaily
\end{verbatim}

\subsection{Recreating a graphic}\label{recreating-a-graphic}

The original data has 18 different entries for the \texttt{group}
variable:

\begin{Shaded}
\begin{Highlighting}[]
\KeywordTok{unique}\NormalTok{(Daily$group)}
\end{Highlighting}
\end{Shaded}

\begin{verbatim}
##  [1] "Acting"         "Comedy"        
##  [3] "Musician"       "Media"         
##  [5] NA               "Politician"    
##  [7] "Athletics"      "Business"      
##  [9] "Advocacy"       "Political Aide"
## [11] "Misc"           "Academic"      
## [13] "Government"     "media"         
## [15] "Clergy"         "Science"       
## [17] "Consultant"     "Military"
\end{verbatim}

In order to help you recreate the first figure from
\href{https://fivethirtyeight.com/datalab/every-guest-jon-stewart-ever-had-on-the-daily-show/}{the
article}, I have added a new variable with three broader groups: (i)
entertainment; (ii) politics, business, and government, and (iii)
commentators. We will learn in the next activity what the
\texttt{inner\_join} in this code chunk is doing.

\begin{Shaded}
\begin{Highlighting}[]
\NormalTok{DailyGroups<-}\KeywordTok{read_csv}\NormalTok{(}\StringTok{"https://www.macalester.edu/~dshuman1/data/112/daily-group-assignment.csv"}\NormalTok{)}
\NormalTok{Daily<-Daily%>%}
\StringTok{  }\KeywordTok{inner_join}\NormalTok{(DailyGroups,}\DataTypeTok{by=}\KeywordTok{c}\NormalTok{(}\StringTok{"group"}\NormalTok{=}\StringTok{"group"}\NormalTok{))}
\end{Highlighting}
\end{Shaded}

\begin{tabular}{r|l|l|l|l|l}
\hline
year & google\_knowledge\_occupation & show & group & raw\_guest\_list & broad\_group\\
\hline
1999 & actor & 1999-01-11 & Acting & Michael J. Fox & Entertainment\\
\hline
1999 & comedian & 1999-01-12 & Comedy & Sandra Bernhard & Entertainment\\
\hline
1999 & television actress & 1999-01-13 & Acting & Tracey Ullman & Entertainment\\
\hline
1999 & film actress & 1999-01-14 & Acting & Gillian Anderson & Entertainment\\
\hline
1999 & actor & 1999-01-18 & Acting & David Alan Grier & Entertainment\\
\hline
1999 & actor & 1999-01-19 & Acting & William Baldwin & Entertainment\\
\hline
\end{tabular}

\BeginKnitrBlock{exercise}
\protect\hypertarget{exr:unnamed-chunk-19}{}{\label{exr:unnamed-chunk-19}
}Using the group assignments contained in the \texttt{broad\_group}
variable, recreate the graphic from the article, with three different
lines showing the fraction of guests in each group over time. Hint:
first think about what your case should be for the glyph-ready form.
\EndKnitrBlock{exercise}

\section{Gathering Practice}\label{gathering-practice}

A typical situation that requires a \texttt{gather} command is when the
columns represent the possible values of a variable. Table
\ref{tab:lesotho-table} shows example data set from
\href{http://dataportal.opendataforafrica.org/}{opendataforafrica.org}
with different years in different columns.

\begin{Shaded}
\begin{Highlighting}[]
\NormalTok{Lesotho<-}\KeywordTok{read_csv}\NormalTok{(}\StringTok{"https://www.macalester.edu/~dshuman1/data/112/Lesotho.csv"}\NormalTok{)}
\end{Highlighting}
\end{Shaded}

\begin{table}

\caption{\label{tab:lesotho-table}Financial statistics about Lesotho.}
\centering
\begin{tabular}[t]{l|r|r|r|r|r}
\hline
Category & 2010 & 2011 & 2012 & 2013 & 2014\\
\hline
Total Population & 2.01 & 2.03 & 2.05 & 2.07 & 2.10\\
\hline
Gross Domestic Product & 2242.30 & 2560.99 & 2494.60 & 2267.96 & 1929.28\\
\hline
Average Interest Rate on Loans & 11.22 & 10.43 & 10.12 & 9.92 & 10.34\\
\hline
Inflation Rate & 3.60 & 4.98 & 6.10 & 5.03 & 4.94\\
\hline
Average Interest Rate on Deposits & 3.68 & 2.69 & 2.85 & 2.85 & 2.73\\
\hline
\end{tabular}
\end{table}

\BeginKnitrBlock{exercise}[Gathering practice]
\protect\hypertarget{exr:unnamed-chunk-21}{}{\label{exr:unnamed-chunk-21}
\iffalse (Gathering practice) \fi{} }Make a side-by-side bar chart with
the \texttt{year} on the horizontal axis, and three side-by-side
vertical columns for average interest rate on deposits, average interest
rate on loans, and inflation rate for each year. In order to get the
data into glyph-ready form, you'll need to use
\texttt{gather}.\footnote{Hint: \texttt{gather} uses the
  \texttt{dplyr::select()} notation, so you can, e.g., list the columns
  you want to select, use colon notation, or use
  \texttt{contains(a\ string)}. See
  \href{http://r4ds.had.co.nz/transform.html\#select-columns-with-select}{Wickham
  and Grolemund} for more information.}
\EndKnitrBlock{exercise}



\end{document}
